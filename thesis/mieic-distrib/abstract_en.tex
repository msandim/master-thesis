\chapter*{Abstract}

Anomaly Detection is an important research topic nowadays, in which the intention is to find patterns in data that do not conform to expected behavior. 
This concept is applicable in a large number of different domains and contexts, such as intrusion detection, fraud detection, medical research and social network analysis.

Techniques that have been addressed within this topic are diverse, based on different assumptions about how anomalies manifest themselves within the data and can have different outputs (i.e. a numeric score or a labeled classification).
Because of this heterogeneity, every technique is specialized in specific characteristics of the data and may only provide a limited insight on what anomalies exist in a given dataset.

Ensemble Learning is process that tries to incorporate the opinions of different learners in order to make a more pondered decision.
This process has been successfully applied in the past to supervised and unsupervised learning problems and improvements in performance have been empirically observed.
Stacked Generalization is one of these methods, in which a learning algorithm is used to combine the different learners.

Several state of the art Anomaly Detection techniques and datasets used throughout the literature were used in this work, which was divided in two different research studies.
The first study focused on the performance and diversity of the Anomaly Detection techniques selected, while the second one focused on the application of Stacked Generalization to the techniques selected.

The first study gathered some evidence that most Anomaly Detection techniques used are \textit{accurate} and \textit{diverse}, therefore allowing the conditions for Stacked Generalization to be applied to this case.
The second study concluded that the Stacked Generalization method guaranteed higher performance than the best Anomaly Detection technique on more than half of the datasets used.
Replacing the Stacked Generalization method's meta-classifier with a simpler Majority Voting method improved the performance on even more datasets.

Possible future work could include gathering datasets with more observations and using a higher variety of Anomaly Detection techniques. This last point would likely require some implementation work, since most of the techniques referred in the literature are not implemented on general purpose programming languages.

\chapter*{Resumo}

Deteção de Anomalias é uma área de investigação importante hoje em dia, na qual a intenção é encontrar padrões em dados que não estejam de acordo com o comportamento esperado.
Este conceito é aplicável a um grande número de diferentes domínios e contextos, como deteção de intrusões, deteção de fraude, investigação médica e análise de redes sociais.

As técnicas que têm sido utilizadas nesta área são diversas, baseadas em diferentes assunções sobre como as anomalias se manifestam nos dados e podem ter diferentes resultados (uma pontuação numérica ou uma classificação). Devido a esta heterogeneidade, cada técnica é especializada em características específicas dos dados e pode apenas fornecer uma visão limitada sobre as anomalias que existem num conjunto de dados específico.

\textit{Ensemble Learning} é um processo que tenta incorporar as opiniões de diferentes algoritmos de modo a potenciar uma decisão mais ponderada.
Este processo tem sido aplicado com sucesso em problemas de aprendizagem supervisionada e não-supervisionada e melhorias na performance foram observadas empiricamente.
\textit{Stacked Generalization} é um destes métodos, no qual um algoritmo de aprendizagem é usado para combinar as opiniões de diferentes algoritmos.

Várias técnicas do estado de arte de Deteção de Anomalias e conjuntos de dados usados na literatura foram usados neste trabalho, que foi dividido em dois diferentes estudos de investigação.
O primeiro estudo focou-se na performance e diversidade das técnicas de Deteção de Anomalias selecionadas, enquanto o segundo focou-se na aplicação de \textit{Stacked Generalization} nas técnicas selecionadas.

O primeiro estudo revelou algumas evidências de que a maioria das técnicas de Deteção de Anomalias usadas é \textit{exata} e \textit{diversa}, garantindo as condições para que o \textit{Stacked Generalization} seja aplicado a este caso.
O segundo estudo concluiu que o método \textit{Stacked Generalization} garantiu uma maior performance que a melhor técnica de Deteção de Anomalias em mais de metade dos conjuntos de dados usados.
Substituindo o meta-classificador do método \textit{Stacked Generalization} por um método \textit{Majority Voting} simples melhorou a performance em ainda mais conjuntos de dados.

Possível trabalho futuro inclui reunir conjuntos de dados com mais observações e usar uma variedade maior de técnicas de Deteção de Anomalias. Este último ponto provavelmente requererá algum trabalho de implementação, dado que a maior das técnicas referidas na literatura não estão implementadas nas linguagens de programação comuns.
