\chapter{Methodology} \label{chap:metho}

\section*{}

This chapter will provide an overview of the methodology to be followed throughout this thesis.

\section{Overview of the Proposed Methodology}

\begin{figure}[!ht]
	\centering
	\includegraphics[width=0.7\textwidth]{methodology.pdf}
	\caption{Scheme representing the proposed methodology to be followed throughout this thesis research work.}
	\label{fig:methodology}
\end{figure}

The proposed methodology to be followed during this thesis is proposed in figure \ref{fig:methodology} and incorporates the following steps:

\begin{enumerate}
	\item Identification of possible Ensemble Learning methods that can be adapted to Anomaly Detection techniques. Methods that are reported in the literature as not being used before in this context will be given special attention.
	
	\item Study of the methods identified and eventual adaptations to be applied to the Anomaly Detection context.
	
	\item Implementation of the adapted methods or use of open-source libraries that already implement them.
	
	\item Validation of the methods with benchmarking on well-known datasets throughout the literature Anomaly Detection. Several different Anomaly Detection techniques will also be used in the ensembles.
	
	\item Analysis of the of the results obtained in the previous phase.
	
	\item Proposal of new possible methods, given the results of the previous phase. Step 2 of the proposed methodology (along with the following steps) will then be repeated.
\end{enumerate}

This methodology will follow two iterations:

\begin{itemize}
	\item In the first iteration a set of Ensemble Learning methods will be proposed (based on the current literature on the topic and identified gaps), implemented and evaluated.
	\item In the second iteration the same procedure will be repeated, except that the Ensemble Learning methods proposed will be based on the results from the first iteration.
\end{itemize}

 

The proposed methods will then be evaluated on datasets from the industry in order to identify possible domains in which these techniques might have better performance.

