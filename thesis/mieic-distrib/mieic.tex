%% FEUP THESIS STYLE for LaTeX2e
%% how to use feupteses (portuguese version)
%%
%% FEUP, JCL & JCF, 31 Jul 2012
%%
%% PLEASE send improvements to jlopes at fe.up.pt and to jcf at fe.up.pt
%%

%%========================================
%% Commands: pdflatex tese
%%           bibtex tese
%%           makeindex tese (only if creating an index) 
%%           pdflatex tese
%% Alternative:
%%          latexmk -pdf tese.tex
%%========================================

\documentclass[11pt,a4paper,twoside,openright]{report}

%% For iso-8859-1 (latin1), comment next line and uncomment the second line
\usepackage[utf8]{inputenc}
%\usepackage[latin1]{inputenc}

%% Portuguese version

%% MIEIC options
\usepackage[portugues,mieic]{feupteses}
%\usepackage[portugues,mieic,juri]{feupteses}
%\usepackage[portugues,mieic,final]{feupteses}
%\usepackage[portugues,mieic,final,onpaper]{feupteses}

%% Options: 
%% - portugues: titles, etc in portuguese
%% - onpaper: links are not shown (for paper versions)
%% - backrefs: include back references from bibliography to citation place

%% Uncomment the next lines if side by side graphics used
%\usepackage[lofdepth,lotdepth]{subfig}
%\usepackage{graphicx}
%\usepackage{float}

%% Include color package
\usepackage{color}
\definecolor{cloudwhite}{cmyk}{0,0,0,0.025}

%% Include source-code listings package
\usepackage{listings}
\lstset{ %
 language=C,                        % choose the language of the code
 basicstyle=\footnotesize\ttfamily,
 keywordstyle=\bfseries,
 numbers=left,                      % where to put the line-numbers
 numberstyle=\scriptsize\texttt,    % the size of the fonts that are used for the line-numbers
 stepnumber=1,                      % the step between two line-numbers. If it's 1 each line will be numbered
 numbersep=8pt,                     % how far the line-numbers are from the code
 frame=tb,
 float=htb,
 aboveskip=8mm,
 belowskip=4mm,
 backgroundcolor=\color{cloudwhite},
 showspaces=false,                  % show spaces adding particular underscores
 showstringspaces=false,            % underline spaces within strings
 showtabs=false,                    % show tabs within strings adding particular underscores
 tabsize=2,	                    % sets default tabsize to 2 spaces
 captionpos=b,                      % sets the caption-position to bottom
 breaklines=true,                   % sets automatic line breaking
 breakatwhitespace=false,           % sets if automatic breaks should only happen at whitespace
 escapeinside={\%*}{*)},            % if you want to add a comment within your code
 morekeywords={*,var,template,new}  % if you want to add more keywords to the set
}

%% Uncomment next line to set the depth of sectional units listed in the toc
%\setcounter{tocdepth}{3}

%% Uncomment to create an index (at the end of the document)
%\makeindex

%% Path to the figures directory
%% TIP: use folder ``figures'' to keep all your figures
\graphicspath{{figures/}}

%%----------------------------------------
%% TIP: if you want to define more macros, use an external file to keep them
%some macro definitions

% For footnotes to work in tables:
\usepackage{footnote}
\makesavenoteenv{table}
\makesavenoteenv{tabular}

% format
\newcommand{\class}[1]{{\normalfont\slshape #1\/}}

% entities
\newcommand{\Feup}{Faculdade de Engenharia da Universidade do Porto}

\newcommand{\svg}{\class{SVG}}
\newcommand{\scada}{\class{SCADA}}
\newcommand{\scadadms}{\class{SCADA/DMS}}

%% For notes:
\usepackage{todonotes}
\newcommand\miguel[1]{\todo[color=green!40]{\small} #1}

% References
\usepackage{csquotes}
\usepackage[style=trad-alpha, backend=biber, maxcitenames=2]{biblatex}
\addbibresource{library.bib}
\addbibresource{web.bib}

\newcommand { \PrintBibliographyBiblatex }
{%
	\renewcommand{\bibname}{References}%
	\cleardoublepage%
	\phantomsection%
	\addcontentsline{toc}{chapter}{References}%
	\begin{singlespace}
		\printbibliography
	\end{singlespace}
}

% Definitions:
\usepackage{amsmath,amssymb}
\DeclareMathAlphabet{\mathcal}{OMS}{cmsy}{m}{n}
\newtheorem{definition}{Definition}[section]

\usepackage{float} % floating elements
\usepackage{booktabs} % tables
\usepackage{multirow} % For the tables

% For floats to not appear after
\usepackage{placeins}

\let\Oldsection\section
\renewcommand{\section}{\FloatBarrier\Oldsection}

\let\Oldsubsection\subsection
\renewcommand{\subsection}{\FloatBarrier\Oldsubsection}

\let\Oldsubsubsection\subsubsection
\renewcommand{\subsubsection}{\FloatBarrier\Oldsubsubsection}
%%----------------------------------------

%%========================================
%% Start of document
%%========================================
\begin{document}

%%----------------------------------------
%% Information about the work
%%----------------------------------------
\title{Título da Dissertação}
\author{Nome do Autor}

%% Uncomment next line for date of submission
%\thesisdate{31 de Julho de 2008}

%% Uncomment next line for copyright text if used
%\copyrightnotice{Nome do Autor, 2008}

\supervisor{Orientador}{Nome do Orientador}

%% Uncomment next line if necessary
%\supervisor{Co-orientador}{Nome de Outro Orientador}

%% Uncomment committee stuff in the final version
%\committeetext{Aprovado em provas públicas pelo Júri:}
%\committeemember{Presidente}{Nome do presidente do júri}
%\committeemember{Arguente}{Nome do arguente do júri}
%\committeemember{Vogal}{Nome do vogal do júri}
%\signature

%% Specify cover logo (in folder ``figures'')
\logo{uporto-feup.pdf}

%% Uncomment next line for additional text below the author's name (front page)
%\additionalfronttext{Preparação da Dissertação}

%%----------------------------------------
%% Preliminary materials
%%----------------------------------------

% remove unnecessary \include{} commands
\begin{Prolog}
  \chapter*{Resumo}

O Resumo fornece ao leitor um sumário do conteúdo da dissertação.
Deverá ser breve mas conter detalhe suficiente e, uma vez que é a porta
de entrada para a dissertação, deverá dar ao leitor uma boa impressão
inicial.

Este texto inicial da dissertação é escrito no fim e resume numa
página, sem referências externas, o tema e o contexto do trabalho, a
motivação e os objectivos, as metodologias e técnicas empregues, os
principais resultados alcançados e as conclusões.

Este documento ilustra o formato a usar em dissertações na \Feup.
São dados exemplos de margens, cabeçalhos, títulos, paginação, estilos
de índices, etc. 
São ainda dados exemplos de formatação de citações, figuras e tabelas,
equações, referências cruzadas, lista de referências e índices.
%Este documento não pretende exemplificar conteúdos a usar. 
É usado texto descartável, \emph{Loren Ipsum}, para preencher a
dissertação por forma a ilustrar os formatos.

Seguem-se umas notas breves mas muito importantes sobre a versão 
provisória e a versão final do documento. 
A versão provisória, depois de verificada pelo orientador e de 
corrigida em contexto pelo autor, deve ser publicada na página 
pessoal de cada estudante/dissertação, juntamente com os dois 
resumos, em português e em inglês; deve manter a marca da água, 
assim como a numeração de linhas conforme aqui se demonstra.

A versão definitiva, a produzir somente após a defesa, em versão 
impressa (dois exemplares com capas próprias FEUP) e em versão 
eletrónica (6 CDs com "rodela" própria FEUP), deve ser limpa da marca de 
água e da numeração de linhas e deve conter a identificação, na primeira 
página, dos elementos do júri respetivo. 
Deve ainda, se for o caso, ser corrigida de acordo com as instruções 
recebidas dos elementos júri.

Lorem ipsum dolor sit amet, consectetuer adipiscing elit. Sed vehicula
lorem commodo dui. Fusce mollis feugiat elit. Cum sociis natoque
penatibus et magnis dis parturient montes, nascetur ridiculus
mus. Donec eu quam. Aenean consectetuer odio quis nisi. Fusce molestie
metus sed neque. Praesent nulla. Donec quis urna. Pellentesque
hendrerit vulputate nunc. Donec id eros et leo ullamcorper
placerat. Curabitur aliquam tellus et diam. 

Ut tortor. Morbi eget elit. Maecenas nec risus. Sed ultricies. Sed
scelerisque libero faucibus sem. Nullam molestie leo quis
tellus. Donec ipsum. Nulla lobortis purus pharetra turpis. Nulla
laoreet, arcu nec hendrerit vulputate, tortor elit eleifend turpis, et
aliquam leo metus in dolor. Praesent sed nulla. Mauris ac augue. Cras
ac orci. Etiam sed urna eget nulla sodales venenatis. Donec faucibus
ante eget dui. Nam magna. Suspendisse sollicitudin est et mi. 

Phasellus ullamcorper justo id risus. Nunc in leo. Mauris auctor
lectus vitae est lacinia egestas. Nulla faucibus erat sit amet lectus
varius semper. Praesent ultrices vehicula orci. Nam at metus. Aenean
eget lorem nec purus feugiat molestie. Phasellus fringilla nulla ac
risus. Aliquam elementum aliquam velit. Aenean nunc odio, lobortis id,
dictum et, rutrum ac, ipsum. 

Ut tortor. Morbi eget elit. Maecenas nec risus. Sed ultricies. Sed
scelerisque libero faucibus sem. Nullam molestie leo quis
tellus. Donec ipsum. 

\chapter*{Abstract}

Here goes the abstract written in English.

Lorem ipsum dolor sit amet, consectetuer adipiscing elit. Sed vehicula
lorem commodo dui. Fusce mollis feugiat elit. Cum sociis natoque
penatibus et magnis dis parturient montes, nascetur ridiculus
mus. Donec eu quam. Aenean consectetuer odio quis nisi. Fusce molestie
metus sed neque. Praesent nulla. Donec quis urna. Pellentesque
hendrerit vulputate nunc. Donec id eros et leo ullamcorper
placerat. Curabitur aliquam tellus et diam. 

Ut tortor. Morbi eget elit. Maecenas nec risus. Sed ultricies. Sed
scelerisque libero faucibus sem. Nullam molestie leo quis
tellus. Donec ipsum. Nulla lobortis purus pharetra turpis. Nulla
laoreet, arcu nec hendrerit vulputate, tortor elit eleifend turpis, et
aliquam leo metus in dolor. Praesent sed nulla. Mauris ac augue. Cras
ac orci. Etiam sed urna eget nulla sodales venenatis. Donec faucibus
ante eget dui. Nam magna. Suspendisse sollicitudin est et mi. 

Fusce sed ipsum vel velit imperdiet dictum. Sed nisi purus, dapibus
ut, iaculis ac, placerat id, purus. Integer aliquet elementum
libero. Phasellus facilisis leo eget elit. Nullam nisi magna, ornare
at, aliquet et, porta id, odio. Sed volutpat tellus consectetuer
ligula. Phasellus turpis augue, malesuada et, placerat fringilla,
ornare nec, eros. Class aptent taciti sociosqu ad litora torquent per
conubia nostra, per inceptos himenaeos. Vivamus ornare quam nec sem
mattis vulputate. Nullam porta, diam nec porta mollis, orci leo
condimentum sapien, quis venenatis mi dolor a metus. Nullam
mollis. Aenean metus massa, pellentesque sit amet, sagittis eget,
tincidunt in, arcu. Vestibulum porta laoreet tortor. Nullam mollis
elit nec justo. In nulla ligula, pellentesque sit amet, consequat sed,
faucibus id, velit. Fusce purus. Quisque sagittis urna at quam. Ut eu
lacus. Maecenas tortor nibh, ultricies nec, vestibulum varius, egestas
id, sapien. 

Phasellus ullamcorper justo id risus. Nunc in leo. Mauris auctor
lectus vitae est lacinia egestas. Nulla faucibus erat sit amet lectus
varius semper. Praesent ultrices vehicula orci. Nam at metus. Aenean
eget lorem nec purus feugiat molestie. Phasellus fringilla nulla ac
risus. Aliquam elementum aliquam velit. Aenean nunc odio, lobortis id,
dictum et, rutrum ac, ipsum. 

Ut tortor. Morbi eget elit. Maecenas nec risus. Sed ultricies. Sed
scelerisque libero faucibus sem. Nullam molestie leo quis
tellus. Donec ipsum. Nulla lobortis purus pharetra turpis. Nulla
laoreet, arcu nec hendrerit vulputate, tortor elit eleifend turpis, et
aliquam leo metus in dolor. Praesent sed nulla. Mauris ac augue. Cras
ac orci. Etiam sed urna eget nulla sodales venenatis. Donec faucibus
ante eget dui. Nam magna. Suspendisse sollicitudin est et mi. 

Phasellus ullamcorper justo id risus. Nunc in leo. Mauris auctor
lectus vitae est lacinia egestas. Nulla faucibus erat sit amet lectus
varius semper. Praesent ultrices vehicula orci. Nam at metus. Aenean
eget lorem nec purus feugiat molestie. Phasellus fringilla nulla ac
risus. Aliquam elementum aliquam velit. Aenean nunc odio, lobortis id,
dictum et, rutrum ac, ipsum. 

Ut tortor. Morbi eget elit. Maecenas nec risus. Sed ultricies. Sed
scelerisque libero faucibus sem. Nullam molestie leo quis
tellus. Donec ipsum. 
 % the abstract
  \chapter*{Agradecimentos}
%\addcontentsline{toc}{chapter}{Agradecimentos}

Aliquam id dui. Nulla facilisi. Nullam ligula nunc, viverra a, iaculis
at, faucibus quis, sapien. Cum sociis natoque penatibus et magnis dis
parturient montes, nascetur ridiculus mus. Curabitur magna ligula,
ornare luctus, aliquam non, aliquet at, tortor. Donec iaculis nulla
sed eros. Sed felis. Nam lobortis libero. Pellentesque
odio. Suspendisse potenti. Morbi imperdiet rhoncus magna. Morbi
vestibulum interdum turpis. Pellentesque varius. Morbi nulla urna,
euismod in, molestie ac, placerat in, orci. 

Ut convallis. Suspendisse luctus pharetra sem. Sed sit amet mi in diam
luctus suscipit. Nulla facilisi. Integer commodo, turpis et semper
auctor, nisl ligula vestibulum erat, sed tempor lacus nibh at
turpis. Quisque vestibulum pulvinar justo. Class aptent taciti
sociosqu ad litora torquent per conubia nostra, per inceptos
himenaeos. Nam sed tellus vel tortor hendrerit pulvinar. Phasellus
eleifend, augue at mattis tincidunt, lorem lorem sodales arcu, id
volutpat risus est id neque. Phasellus egestas ante. Nam porttitor
justo sit amet urna. Suspendisse ligula nunc, mollis ac, elementum
non, venenatis ut, mauris. Mauris augue risus, tempus scelerisque,
rutrum quis, hendrerit at, nunc. Nulla posuere porta orci. Nulla dui. 

Fusce gravida placerat sem. Aenean ipsum diam, pharetra vitae, ornare
et, semper sit amet, nibh. Nam id tellus. Etiam ultrices. Praesent
gravida. Aliquam nec sapien. Morbi sagittis vulputate dolor. Donec
sapien lorem, laoreet egestas, pellentesque euismod, porta at,
sapien. Integer vitae lacus id dui convallis blandit. Mauris non
sem. Integer in velit eget lorem scelerisque vehicula. Etiam tincidunt
turpis ac nunc. Pellentesque a justo. Mauris faucibus quam id
eros. Cras pharetra. Fusce rutrum vulputate lorem. Cras pretium magna
in nisl. Integer ornare dui non pede. 

\vspace{10mm}
\flushleft{O Nome do Autor}
  % the acknowledgments
  \cleardoublepage
\thispagestyle{plain}

\vspace*{8cm}

%\begin{flushright}
%   \textsl{``You should be glad that bridge fell down. \\
%           I was planning to build thirteen more to that same design''} \\
%\vspace*{1.5cm}
%           Isambard Kingdom Brunel
%\end{flushright}

%\begin{flushright}
%   \textsl{Follow your passion. \\
%   		   Stay true to yourself.\\
%   		   Never follow someone else's path unless you're in the woods \\
%   		   and you're lost and you see a path.\\
%   		   By all means, you should follow that.} \\
%\vspace*{1.5cm}
%           Ellen DeGeneres
%\end{flushright}

\begin{flushright}
   \textsl{``Your time is limited, so don't waste it living someone else's life.\\
   	Don't be trapped by dogma -- which is living with the results of other people's thinking.\\
   	Don't let the noise of others' opinions drown out your own inner voice.\\
   	And most important, have the courage to follow your heart and intuition.''} \\
\vspace*{1.5cm}
           Steve Jobs
\end{flushright}




    % initial quotation if desired
  \cleardoublepage
  \pdfbookmark[0]{Conteúdo}{contents}
  \tableofcontents
  \cleardoublepage
  \pdfbookmark[0]{Lista de Figuras}{figures}
  \listoffigures
  \cleardoublepage
  \pdfbookmark[0]{Lista de Tabelas}{tables}
  \listoftables
  \chapter*{Abreviaturas e Símbolos}
%\addcontentsline{toc}{chapter}{Abbreviations}
\chaptermark{ABREVIATURAS E SÍMBOLOS}

\begin{flushleft}
\begin{tabular}{l p{0.8\linewidth}}
ADT      & Abstract Data Type\\
ANDF     & Architecture-Neutral Distribution Format\\
API      & Application Programming Interface\\
CAD      & Computer-Aided Design\\
CASE     & Computer-Aided Software Engineering\\
CORBA    & Common Object Request Broker Architecture\\
UNCOL    & UNiversal COmpiler-oriented Language\\
Loren    & Lorem ipsum dolor sit amet, consectetuer adipiscing
elit. Sed vehicula lorem commodo dui\\
WWW      & \emph{World Wide Web}
\end{tabular}
\end{flushleft}

  % the list of abbreviations used
\end{Prolog}

%%----------------------------------------
%% Body
%%----------------------------------------

\StartBody

%% TIP: use a separate file for each chapter
\include{chapter1} 
\include{chapter2}
\chapter{Methodology} \label{chap:metho}

\section*{}

This chapter will provide an overview of the methodology to be followed throughout this thesis.

\section{Overview of the Proposed Methodology}

\begin{figure}[!ht]
	\centering
	\includegraphics[width=0.7\textwidth]{methodology.pdf}
	\caption{Scheme representing the proposed methodology to be followed throughout this thesis research work.}
	\label{fig:methodology}
\end{figure}

The proposed methodology to be followed during this thesis is proposed in figure \ref{fig:methodology} and incorporates the following steps:

\begin{enumerate}
	\item Identification of possible Ensemble Learning methods that can be adapted to Anomaly Detection techniques. Methods that are reported in the literature as not being used before in this context will be given special attention.
	
	\item Study of the methods identified and eventual adaptations to be applied to the Anomaly Detection context.
	
	\item Implementation of the adapted methods or use of open-source libraries that already implement them.
	
	\item Validation of the methods with benchmarking on well-known datasets throughout the literature Anomaly Detection. Several different Anomaly Detection techniques will also be used in the ensembles.
	
	\item Analysis of the of the results obtained in the previous phase.
	
	\item Proposal of new possible methods, given the results of the previous phase. Step 2 of the proposed methodology (along with the following steps) will then be repeated.
\end{enumerate}

This methodology will follow two iterations:

\begin{itemize}
	\item In the first iteration a set of Ensemble Learning methods will be proposed (based on the current literature on the topic and identified gaps), implemented and evaluated.
	\item In the second iteration the same procedure will be repeated, except that the Ensemble Learning methods proposed will be based on the results from the first iteration.
\end{itemize}

 

The proposed methods will then be evaluated on datasets from the industry in order to identify possible domains in which these techniques might have better performance.


\chapter{Conclusions and Future Work} \label{chap:conc}

\section*{}

This chapter will present preliminary conclusions of the work developed so far, as well as the future work planning.

\section{Conclusions}

The work developed so far allowed to conclude that the application of Ensemble Learning to Anomaly Detection is a novel research topic within Data Mining and that this topic reveals potential for improvement. 

However some difficulties are also reported throughout the literature \cite{Aggarwal2013, Zimek2014}, the most critical being the difficulty to adapt known Ensemble Learning methods to unsupervised learning (which is reported as being the most common learning mode in Anomaly Detection).

\section{Future Work}

As proposed in chapter \ref{chap:metho}, this thesis will follow a two-iteration methodology. Figure \ref{fig:gantt} illustrates the planning for the future work as a gantt chart.

\begin{figure}[!ht]
	\centering
	\includegraphics[width=1.\textwidth]{future_work_plan.png}
	\caption{Gantt chart with the main phases of the future work to be developed.}
	\label{fig:gantt}
\end{figure}

This planning will be described in detail on a weekly basis in this section.

\subsection{February 20 - February 24}

\begin{itemize}
	\item Writing of the remaining related work in Ensemble Learning and its applications in the Anomaly Detection context.
	\item Identification of methods present in the Ensemble Learning literature that were not applied to Anomaly Detection techniques yet.
\end{itemize}

\subsection{February 27 - March 3}

\begin{itemize}
	\item Conclusion of the related work writing.
	\item Study of the possible application of the identified Ensemble Learning methods in the context of Anomaly Detection. Identification of eventual adaptations of these methods.
\end{itemize}

\subsection{March 6 - March 10}

\begin{itemize}
	\item Conclusion of the study of the possible methods and eventual adaptations.
	\item Research on possible libraries that implement the selected methods.
\end{itemize}

\subsection{March 13 - March 107}

\begin{itemize}
	\item Implementation of the selected methods.
\end{itemize}

\subsection{March 20 - March 24}

\begin{itemize}
	\item Implementation of the selected methods.
\end{itemize}

\subsection{March 27 - March 31}

\begin{itemize}
	\item Implementation of the selected methods.
\end{itemize}

\subsection{April 3 - April 7}

\begin{itemize}
	\item Benchmark of the implemented methods on a set of Anomaly Detection datasets.
	\item Reporting of methods implemented and the obtained results.
	\item Evaluation of the implemented methods, by comparing the results with the results of the existing Ensemble Learning approaches in Anomaly Detection.
\end{itemize}

\subsection{April 10 - April 14}

\begin{itemize}
	\item Study on possible modifications of the implemented methods.
	\item Study on possible new methods that can be implemented.
\end{itemize}

\subsection{April 17 - April 21}

\begin{itemize}
	\item Conclusion of the tasks from the previous week.
\end{itemize}

\subsection{April 24 - April 28}

\begin{itemize}
	\item Research on possible libraries that implement the selected methods.
	\item Implementation of the selected methods.
\end{itemize}

\subsection{May 1 - May 5}

\begin{itemize}
	\item Implementation of the selected methods.
\end{itemize}

\subsection{May 8 - May 12}

\begin{itemize}
	\item Benchmark of the implemented methods on a set of Anomaly Detection datasets.
	\item Reporting of methods implemented and the obtained results.
	\item Evaluation of the implemented methods, by comparing the results with the results of the existing Ensemble Learning approaches in Anomaly Detection and with the results obtained the previous evaluation phase.
\end{itemize}

\subsection{May 15 - June 23}

\begin{itemize}
	\item Final reporting of the methodology used and analysis of the results obtained.
	\item Preparation of the final presentation.
\end{itemize}
\include{chapter5} 

%%----------------------------------------
%% Final materials
%%----------------------------------------

%% Bibliography
%% Comment the next command if BibTeX file not used, 
%% Assumes that bibliography is in ``myrefs.bib''
\PrintBib{myrefs}

%% Comment next 2 commands if numbered appendices are not used
\appendix
\chapter{Loren Ipsum} \label{ap1:loren}

Depois das conclusões e antes das referências bibliográficas,
apresenta-se neste anexo numerado o texto usado para preencher a
dissertação.

\section{O que é o \emph{Loren Ipsum}?}

\emph{\textbf{Lorem Ipsum}} is simply dummy text of the printing and
typesetting industry. Lorem Ipsum has been the industry's standard
dummy text ever since the 1500s, when an unknown printer took a galley
of type and scrambled it to make a type specimen book. It has survived
not only five centuries, but also the leap into electronic
typesetting, remaining essentially unchanged. It was popularised in
the 1960s with the release of Letraset sheets containing Lorem Ipsum
passages, and more recently with desktop publishing software like
Aldus PageMaker including versions of Lorem Ipsum. 

\section{De onde Vem o Loren?}

Contrary to popular belief, Lorem Ipsum is not simply random text. It
has roots in a piece of classical Latin literature from 45 BC, making
it over 2000 years old. Richard McClintock, a Latin professor at
Hampden-Sydney College in Virginia, looked up one of the more obscure
Latin words, consectetur, from a Lorem Ipsum passage, and going
through the cites of the word in classical literature, discovered the
undoubtable source. Lorem Ipsum comes from sections 1.10.32 and
1.10.33 of ``de Finibus Bonorum et Malorum'' (The Extremes of Good and
Evil) by Cicero, written in 45 BC. This book is a treatise on the
theory of ethics, very popular during the Renaissance. The first line
of Lorem Ipsum, ``Lorem ipsum dolor sit amet\ldots'', comes from a line in
section 1.10.32.

The standard chunk of Lorem Ipsum used since the 1500s is reproduced
below for those interested. Sections 1.10.32 and 1.10.33 from ``de
Finibus Bonorum et Malorum'' by Cicero are also reproduced in their
exact original form, accompanied by English versions from the 1914
translation by H. Rackham.

\section{Porque se usa o Loren?}

It is a long established fact that a reader will be distracted by the
readable content of a page when looking at its layout. The point of
using Lorem Ipsum is that it has a more-or-less normal distribution of
letters, as opposed to using ``Content here, content here'', making it
look like readable English. Many desktop publishing packages and web
page editors now use Lorem Ipsum as their default model text, and a
search for ``lorem ipsum'' will uncover many web sites still in their
infancy. Various versions have evolved over the years, sometimes by
accident, sometimes on purpose (injected humour and the like). 

\section{Onde se Podem Encontrar Exemplos?}

There are many variations of passages of Lorem Ipsum available, but
the majority have suffered alteration in some form, by injected
humour, or randomised words which don't look even slightly
believable. If you are going to use a passage of Lorem Ipsum, you need
to be sure there isn't anything embarrassing hidden in the middle of
text. All the Lorem Ipsum generators on the Internet tend to repeat
predefined chunks as necessary, making this the first true generator
on the Internet. It uses a dictionary of over 200 Latin words,
combined with a handful of model sentence structures, to generate
Lorem Ipsum which looks reasonable. The generated Lorem Ipsum is
therefore always free from repetition, injected humour, or
non-characteristic words etc. 


%% Index
%% Uncomment next command if index is required, 
%% don't forget to run ``makeindex mieic'' command
%\PrintIndex

\end{document}
