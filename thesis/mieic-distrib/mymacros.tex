%some macro definitions

% For footnotes to work in tables:
\usepackage{footnote}
\makesavenoteenv{table}
\makesavenoteenv{tabular}

% format
\newcommand{\class}[1]{{\normalfont\slshape #1\/}}

% entities
\newcommand{\Feup}{Faculdade de Engenharia da Universidade do Porto}

\newcommand{\svg}{\class{SVG}}
\newcommand{\scada}{\class{SCADA}}
\newcommand{\scadadms}{\class{SCADA/DMS}}

%% For notes:
\usepackage{todonotes}
\newcommand\miguel[1]{\todo[color=green!40]{\small} #1}

% References
\usepackage{csquotes}
\usepackage[style=trad-alpha, backend=biber, maxcitenames=2]{biblatex}
\addbibresource{library.bib}
\addbibresource{web.bib}

\newcommand { \PrintBibliographyBiblatex }
{%
	\renewcommand{\bibname}{References}%
	\cleardoublepage%
	\phantomsection%
	\addcontentsline{toc}{chapter}{References}%
	\begin{singlespace}
		\printbibliography
	\end{singlespace}
}

% Definitions:
\usepackage{amsmath,amssymb}
\DeclareMathAlphabet{\mathcal}{OMS}{cmsy}{m}{n}
\newtheorem{definition}{Definition}[section]

\usepackage{float} % floating elements
\usepackage{booktabs} % tables
\usepackage{multirow} % For the tables

% For floats to not appear after
\usepackage{placeins}

\let\Oldsection\section
\renewcommand{\section}{\FloatBarrier\Oldsection}

\let\Oldsubsection\subsection
\renewcommand{\subsection}{\FloatBarrier\Oldsubsection}

\let\Oldsubsubsection\subsubsection
\renewcommand{\subsubsection}{\FloatBarrier\Oldsubsubsection}