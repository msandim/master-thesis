%% FEUP THESIS STYLE for LaTeX2e
%% how to use feupteses (English version)
%%
%% FEUP, JCL & JCF, 31 July 2012
%%
%% PLEASE send improvements to jlopes at fe.up.pt and to jcf at fe.up.pt
%%

%%========================================
%% Commands: pdflatex tese
%%           bibtex tese
%%           makeindex tese (only if creating an index)
%%           pdflatex tese
%% Alternative:
%%          latexmk -pdf tese.tex
%%========================================

\documentclass[11pt,a4paper,twoside,openright]{report}

%% For iso-8859-1 (latin1), comment next line and uncomment the second line
\usepackage[utf8]{inputenc}
%\usepackage[latin1]{inputenc}

%% English version

%% MIEIC options
\usepackage[mieic]{feupteses}
%\usepackage[mieic,juri]{feupteses}
%\usepackage[mieic,final]{feupteses}
%\usepackage[mieic,final,onpaper]{feupteses}

%% Additional options for feupteses.sty: 
%% - onpaper: links are not shown (for paper versions)
%% - backrefs: include back references from bibliography to citation place

%% Uncomment the next lines if side by side graphics used
%\usepackage[lofdepth,lotdepth]{subfig}
%\usepackage{graphicx}
%\usepackage{float}

%% Include color package
\usepackage{color}
\definecolor{cloudwhite}{cmyk}{0,0,0,0.025}

%% Include source-code listings package
\usepackage{listings}
\lstset{ %
 language=C,                        % choose the language of the code
 basicstyle=\footnotesize\ttfamily,
 keywordstyle=\bfseries,
 numbers=left,                      % where to put the line-numbers
 numberstyle=\scriptsize\texttt,    % the size of the fonts that are used for the line-numbers
 stepnumber=1,                      % the step between two line-numbers. If it's 1 each line will be numbered
 numbersep=8pt,                     % how far the line-numbers are from the code
 frame=tb,
 float=htb,
 aboveskip=8mm,
 belowskip=4mm,
 backgroundcolor=\color{cloudwhite},
 showspaces=false,                  % show spaces adding particular underscores
 showstringspaces=false,            % underline spaces within strings
 showtabs=false,                    % show tabs within strings adding particular underscores
 tabsize=2,	                    % sets default tabsize to 2 spaces
 captionpos=b,                      % sets the caption-position to bottom
 breaklines=true,                   % sets automatic line breaking
 breakatwhitespace=false,           % sets if automatic breaks should only happen at whitespace
 escapeinside={\%*}{*)},            % if you want to add a comment within your code
 morekeywords={*,var,template,new}  % if you want to add more keywords to the set
}

%% Uncomment to create an index (at the end of the document)
%\makeindex

%% Path to the figures directory
%% TIP: use folder ``figures'' to keep all your figures
\graphicspath{{figures/}}

%%----------------------------------------
%% TIP: if you want to define more macros, use an external file to keep them
%some macro definitions

% For footnotes to work in tables:
\usepackage{footnote}
\makesavenoteenv{table}
\makesavenoteenv{tabular}

% format
\newcommand{\class}[1]{{\normalfont\slshape #1\/}}

% entities
\newcommand{\Feup}{Faculdade de Engenharia da Universidade do Porto}

\newcommand{\svg}{\class{SVG}}
\newcommand{\scada}{\class{SCADA}}
\newcommand{\scadadms}{\class{SCADA/DMS}}

%% For notes:
\usepackage{todonotes}
\newcommand\miguel[1]{\todo[color=green!40]{\small} #1}

% References
\usepackage{csquotes}
\usepackage[style=trad-alpha, backend=biber, maxcitenames=2]{biblatex}
\addbibresource{library.bib}
\addbibresource{web.bib}

\newcommand { \PrintBibliographyBiblatex }
{%
	\renewcommand{\bibname}{References}%
	\cleardoublepage%
	\phantomsection%
	\addcontentsline{toc}{chapter}{References}%
	\begin{singlespace}
		\printbibliography
	\end{singlespace}
}

% Definitions:
\usepackage{amsmath,amssymb}
\DeclareMathAlphabet{\mathcal}{OMS}{cmsy}{m}{n}
\newtheorem{definition}{Definition}[section]

\usepackage{float} % floating elements
\usepackage{booktabs} % tables
\usepackage{multirow} % For the tables

% For floats to not appear after
\usepackage{placeins}

\let\Oldsection\section
\renewcommand{\section}{\FloatBarrier\Oldsection}

\let\Oldsubsection\subsection
\renewcommand{\subsection}{\FloatBarrier\Oldsubsection}

\let\Oldsubsubsection\subsubsection
\renewcommand{\subsubsection}{\FloatBarrier\Oldsubsubsection}
%%----------------------------------------

%%========================================
%% Start of document
%%========================================
\begin{document}

%%----------------------------------------
%% Information about the work
%%----------------------------------------
\title{Using Ensemble Learning for Anomaly Detection}
\author{Miguel Oliveira Sandim}

%% Uncomment next line for date of submission
%\thesisdate{July 31, 2008}

%%Uncomment next line for copyright text if used
%\copyrightnotice{Name of the Author, 2008}

\supervisor{Supervisor}{Carlos Manuel Milheiro de Oliveira Pinto Soares (Ph.D.)}

%% Uncomment next line if necessary
%\supervisor{Second Supervisor}{Name of the Supervisor}

%% Uncomment committee stuff in the final version if used
%\committeetext{Approved in oral examination by the committee:}
%\committeemember{Chair}{Doctor Name of the President}
%\committeemember{External Examiner}{Doctor Name of the Examiner}
%\committeemember{Supervisor}{Doctor Name of the Supervisor}
%\signature

%% Specify cover logo (in folder ``figures'')
\logo{uporto-feup.pdf}

%% Uncomment next line for additional text  below the author's name (front page)
\additionalfronttext{Dissertation Planning}

%%----------------------------------------
%% Preliminary materials
%%----------------------------------------

% remove unnecssary \include{} commands
\begin{Prolog}
  \chapter*{Abstract}

Anomaly Detection is an important research topic nowadays, in which the intention is to find patterns in data that do not conform to expected behavior. 
This concept is applicable in a large number of different domains and contexts, such as intrusion detection, fraud detection, medical research and social network analysis.

Techniques that have been addressed within this topic are diverse, based on different assumptions about how anomalies manifest themselves within the data and can have different outputs (i.e. a numeric score or a labeled classification).
Because of this heterogeneity, every technique is specialized in specific characteristics of the data and may only provide a limited insight on what anomalies exist in a given dataset.

Ensemble Learning is process that tries to incorporate the opinions of different learners in order to make a more pondered decision.
This process has been successfully applied in the past to supervised and unsupervised learning problems and improvements in performance have been empirically observed.
Stacked Generalization is one of these methods, in which a learning algorithm is used to combine the different learners.

Several state of the art Anomaly Detection techniques and datasets used throughout the literature were used in this work, which was divided in two different research studies.
The first study focused on the performance and diversity of the Anomaly Detection techniques selected, while the second one focused on the application of Stacked Generalization to the techniques selected.

The first study gathered some evidence that most Anomaly Detection techniques used are \textit{accurate} and \textit{diverse}, therefore allowing the conditions for Stacked Generalization to be applied to this case.
The second study concluded that the Stacked Generalization method guaranteed higher performance than the best Anomaly Detection technique on more than half of the datasets used.
Replacing the Stacked Generalization method's meta-classifier with a simpler Majority Voting method improved the performance on even more datasets.

Possible future work could include gathering datasets with more observations and using a higher variety of Anomaly Detection techniques. This last point would likely require some implementation work, since most of the techniques referred in the literature are not implemented on general purpose programming languages.

\chapter*{Resumo}

Deteção de Anomalias é uma área de investigação importante hoje em dia, na qual a intenção é encontrar padrões em dados que não estejam de acordo com o comportamento esperado.
Este conceito é aplicável a um grande número de diferentes domínios e contextos, como deteção de intrusões, deteção de fraude, investigação médica e análise de redes sociais.

As técnicas que têm sido utilizadas nesta área são diversas, baseadas em diferentes assunções sobre como as anomalias se manifestam nos dados e podem ter diferentes resultados (uma pontuação numérica ou uma classificação). Devido a esta heterogeneidade, cada técnica é especializada em características específicas dos dados e pode apenas fornecer uma visão limitada sobre as anomalias que existem num conjunto de dados específico.

\textit{Ensemble Learning} é um processo que tenta incorporar as opiniões de diferentes algoritmos de modo a potenciar uma decisão mais ponderada.
Este processo tem sido aplicado com sucesso em problemas de aprendizagem supervisionada e não-supervisionada e melhorias na performance foram observadas empiricamente.
\textit{Stacked Generalization} é um destes métodos, no qual um algoritmo de aprendizagem é usado para combinar as opiniões de diferentes algoritmos.

Várias técnicas do estado de arte de Deteção de Anomalias e conjuntos de dados usados na literatura foram usados neste trabalho, que foi dividido em dois diferentes estudos de investigação.
O primeiro estudo focou-se na performance e diversidade das técnicas de Deteção de Anomalias selecionadas, enquanto o segundo focou-se na aplicação de \textit{Stacked Generalization} nas técnicas selecionadas.

O primeiro estudo revelou algumas evidências de que a maioria das técnicas de Deteção de Anomalias usadas é \textit{exata} e \textit{diversa}, garantindo as condições para que o \textit{Stacked Generalization} seja aplicado a este caso.
O segundo estudo concluiu que o método \textit{Stacked Generalization} garantiu uma maior performance que a melhor técnica de Deteção de Anomalias em mais de metade dos conjuntos de dados usados.
Substituindo o meta-classificador do método \textit{Stacked Generalization} por um método \textit{Majority Voting} simples melhorou a performance em ainda mais conjuntos de dados.

Possível trabalho futuro inclui reunir conjuntos de dados com mais observações e usar uma variedade maior de técnicas de Deteção de Anomalias. Este último ponto provavelmente requererá algum trabalho de implementação, dado que a maior das técnicas referidas na literatura não estão implementadas nas linguagens de programação comuns.
 % the abstract
  \chapter*{Acknowledgements}

First of all I would like to thank my supervisors, Dr. Carlos Soares and Dr. Bernhard Pfahringer, for guiding me throughout this dissertation topic and pointing me in the right directions when developing this research work.

I would like to thank Daniel for all the support given and kind words in the right moments, my family and my friends João, Ana, Paula, Luís, Susana and Raquel for all their support.
Without them, I know that I would not have been as successful as I was on my research.

Finally, I would like to thank Cláudio Sá, Tiago Cunha, Pedro Ribeiro, Fábio Pinto and Pedro Abreu from INESC TEC for their valuable insights, brainstorming sessions for this dissertation and kind availability to help me.


\vspace{10mm}
\flushleft{Miguel Oliveira Sandim}
  % the acknowledgments
  \cleardoublepage
\thispagestyle{plain}

\vspace*{8cm}

%\begin{flushright}
%   \textsl{``You should be glad that bridge fell down. \\
%           I was planning to build thirteen more to that same design''} \\
%\vspace*{1.5cm}
%           Isambard Kingdom Brunel
%\end{flushright}

%\begin{flushright}
%   \textsl{Follow your passion. \\
%   		   Stay true to yourself.\\
%   		   Never follow someone else's path unless you're in the woods \\
%   		   and you're lost and you see a path.\\
%   		   By all means, you should follow that.} \\
%\vspace*{1.5cm}
%           Ellen DeGeneres
%\end{flushright}

\begin{flushright}
   \textsl{``Your time is limited, so don't waste it living someone else's life.\\
   	Don't be trapped by dogma -- which is living with the results of other people's thinking.\\
   	Don't let the noise of others' opinions drown out your own inner voice.\\
   	And most important, have the courage to follow your heart and intuition.''} \\
\vspace*{1.5cm}
           Steve Jobs
\end{flushright}




       % initial quotation if desired
  \cleardoublepage
  \pdfbookmark[0]{Table of Contents}{contents}
  \tableofcontents
  \cleardoublepage
  \pdfbookmark[0]{List of Figures}{figures}
  \listoffigures
  \cleardoublepage
  \pdfbookmark[0]{List of Tables}{tables}
  \listoftables
  \chapter*{Abbreviations}
\chaptermark{ABBREVIATIONS}

\begin{flushleft}
\begin{tabular}{l p{0.8\linewidth}}
	
SVM & Support Vector Machine \\
LOF & Local Outlier Factor \\
COF & Connectivity-based Outlier Factor \\
ODIN & Outlier Detection using In-degree Number \\
LOCI & Location Correlation Integral \\
DBSCAN & Density-based Spatial Clustering of Applications with Noise \\
SOM & Self Organizing Maps \\
PCA & Principal Component Analysis \\
CAR & Classification Association Rules \\
EM & Expectaction-Maximization \\
CART & Classification And Regression Tree \\
NB & Naive Bayes \\
RF & Random Forest \\
MLP & Multilayer Perceptron \\
LR & Logistic Regression

%ADT      & Abstract Data Type\\
%ANDF     & Architecture-Neutral Distribution Format\\
%API      & Application Programming Interface\\
%CAD      & Computer-Aided Design\\
%CASE     & Computer-Aided Software Engineering\\
%CORBA    & Common Object Request Broker Architecture\\
%UNCOL    & UNiversal COmpiler-oriented Language\\
%Loren    & Lorem ipsum dolor sit amet, consectetuer adipiscing
%elit. Sed vehicula lorem commodo dui\\
%WWW      & \emph{World Wide Web}
\end{tabular}
\end{flushleft}

  % the list of abbreviations used
\end{Prolog}

%%----------------------------------------
%% Body
%%----------------------------------------
\StartBody

%% TIP: use a separate file for each chapter
\include{chapter1} 
\include{chapter2}
\chapter{Methodology} \label{chap:metho}

\section*{}

This chapter will provide an overview of the methodology to be followed throughout this thesis.

\section{Overview of the Proposed Methodology}

\begin{figure}[!ht]
	\centering
	\includegraphics[width=0.7\textwidth]{methodology.pdf}
	\caption{Scheme representing the proposed methodology to be followed throughout this thesis research work.}
	\label{fig:methodology}
\end{figure}

The proposed methodology to be followed during this thesis is proposed in figure \ref{fig:methodology} and incorporates the following steps:

\begin{enumerate}
	\item Identification of possible Ensemble Learning methods that can be adapted to Anomaly Detection techniques. Methods that are reported in the literature as not being used before in this context will be given special attention.
	
	\item Study of the methods identified and eventual adaptations to be applied to the Anomaly Detection context.
	
	\item Implementation of the adapted methods or use of open-source libraries that already implement them.
	
	\item Validation of the methods with benchmarking on well-known datasets throughout the literature Anomaly Detection. Several different Anomaly Detection techniques will also be used in the ensembles.
	
	\item Analysis of the of the results obtained in the previous phase.
	
	\item Proposal of new possible methods, given the results of the previous phase. Step 2 of the proposed methodology (along with the following steps) will then be repeated.
\end{enumerate}

This methodology will follow two iterations:

\begin{itemize}
	\item In the first iteration a set of Ensemble Learning methods will be proposed (based on the current literature on the topic and identified gaps), implemented and evaluated.
	\item In the second iteration the same procedure will be repeated, except that the Ensemble Learning methods proposed will be based on the results from the first iteration.
\end{itemize}

 

The proposed methods will then be evaluated on datasets from the industry in order to identify possible domains in which these techniques might have better performance.


\chapter{Conclusions and Future Work} \label{chap:conc}

\section*{}

This chapter will present preliminary conclusions of the work developed so far, as well as the future work planning.

\section{Conclusions}

The work developed so far allowed to conclude that the application of Ensemble Learning to Anomaly Detection is a novel research topic within Data Mining and that this topic reveals potential for improvement. 

However some difficulties are also reported throughout the literature \cite{Aggarwal2013, Zimek2014}, the most critical being the difficulty to adapt known Ensemble Learning methods to unsupervised learning (which is reported as being the most common learning mode in Anomaly Detection).

\section{Future Work}

As proposed in chapter \ref{chap:metho}, this thesis will follow a two-iteration methodology. Figure \ref{fig:gantt} illustrates the planning for the future work as a gantt chart.

\begin{figure}[!ht]
	\centering
	\includegraphics[width=1.\textwidth]{future_work_plan.png}
	\caption{Gantt chart with the main phases of the future work to be developed.}
	\label{fig:gantt}
\end{figure}

This planning will be described in detail on a weekly basis in this section.

\subsection{February 20 - February 24}

\begin{itemize}
	\item Writing of the remaining related work in Ensemble Learning and its applications in the Anomaly Detection context.
	\item Identification of methods present in the Ensemble Learning literature that were not applied to Anomaly Detection techniques yet.
\end{itemize}

\subsection{February 27 - March 3}

\begin{itemize}
	\item Conclusion of the related work writing.
	\item Study of the possible application of the identified Ensemble Learning methods in the context of Anomaly Detection. Identification of eventual adaptations of these methods.
\end{itemize}

\subsection{March 6 - March 10}

\begin{itemize}
	\item Conclusion of the study of the possible methods and eventual adaptations.
	\item Research on possible libraries that implement the selected methods.
\end{itemize}

\subsection{March 13 - March 107}

\begin{itemize}
	\item Implementation of the selected methods.
\end{itemize}

\subsection{March 20 - March 24}

\begin{itemize}
	\item Implementation of the selected methods.
\end{itemize}

\subsection{March 27 - March 31}

\begin{itemize}
	\item Implementation of the selected methods.
\end{itemize}

\subsection{April 3 - April 7}

\begin{itemize}
	\item Benchmark of the implemented methods on a set of Anomaly Detection datasets.
	\item Reporting of methods implemented and the obtained results.
	\item Evaluation of the implemented methods, by comparing the results with the results of the existing Ensemble Learning approaches in Anomaly Detection.
\end{itemize}

\subsection{April 10 - April 14}

\begin{itemize}
	\item Study on possible modifications of the implemented methods.
	\item Study on possible new methods that can be implemented.
\end{itemize}

\subsection{April 17 - April 21}

\begin{itemize}
	\item Conclusion of the tasks from the previous week.
\end{itemize}

\subsection{April 24 - April 28}

\begin{itemize}
	\item Research on possible libraries that implement the selected methods.
	\item Implementation of the selected methods.
\end{itemize}

\subsection{May 1 - May 5}

\begin{itemize}
	\item Implementation of the selected methods.
\end{itemize}

\subsection{May 8 - May 12}

\begin{itemize}
	\item Benchmark of the implemented methods on a set of Anomaly Detection datasets.
	\item Reporting of methods implemented and the obtained results.
	\item Evaluation of the implemented methods, by comparing the results with the results of the existing Ensemble Learning approaches in Anomaly Detection and with the results obtained the previous evaluation phase.
\end{itemize}

\subsection{May 15 - June 23}

\begin{itemize}
	\item Final reporting of the methodology used and analysis of the results obtained.
	\item Preparation of the final presentation.
\end{itemize}
%\chapter{Visualização de Sinópticos SVG}\label{chap:chap3}

\section*{}
%\chapter{Implementação}\label{chap:chap4}

\section*{}

%\chapter{Conclusões e Trabalho Futuro} \label{chap:concl}

\section*{}
 

%%----------------------------------------
%% Final materials
%%----------------------------------------

%% Bibliography
%% Comment the next command if BibTeX file not used
%% bibliography is in ``myrefs.bib''
%\PrintBib{library}
\PrintBibliographyBiblatex

%% comment next 2 commands if numbered appendices are not used
\appendix
%\chapter{Loren Ipsum} \label{ap1:loren}

Depois das conclusões e antes das referências bibliográficas,
apresenta-se neste anexo numerado o texto usado para preencher a
dissertação.

\section{O que é o \emph{Loren Ipsum}?}

\emph{\textbf{Lorem Ipsum}} is simply dummy text of the printing and
typesetting industry. Lorem Ipsum has been the industry's standard
dummy text ever since the 1500s, when an unknown printer took a galley
of type and scrambled it to make a type specimen book. It has survived
not only five centuries, but also the leap into electronic
typesetting, remaining essentially unchanged. It was popularised in
the 1960s with the release of Letraset sheets containing Lorem Ipsum
passages, and more recently with desktop publishing software like
Aldus PageMaker including versions of Lorem Ipsum. 

\section{De onde Vem o Loren?}

Contrary to popular belief, Lorem Ipsum is not simply random text. It
has roots in a piece of classical Latin literature from 45 BC, making
it over 2000 years old. Richard McClintock, a Latin professor at
Hampden-Sydney College in Virginia, looked up one of the more obscure
Latin words, consectetur, from a Lorem Ipsum passage, and going
through the cites of the word in classical literature, discovered the
undoubtable source. Lorem Ipsum comes from sections 1.10.32 and
1.10.33 of ``de Finibus Bonorum et Malorum'' (The Extremes of Good and
Evil) by Cicero, written in 45 BC. This book is a treatise on the
theory of ethics, very popular during the Renaissance. The first line
of Lorem Ipsum, ``Lorem ipsum dolor sit amet\ldots'', comes from a line in
section 1.10.32.

The standard chunk of Lorem Ipsum used since the 1500s is reproduced
below for those interested. Sections 1.10.32 and 1.10.33 from ``de
Finibus Bonorum et Malorum'' by Cicero are also reproduced in their
exact original form, accompanied by English versions from the 1914
translation by H. Rackham.

\section{Porque se usa o Loren?}

It is a long established fact that a reader will be distracted by the
readable content of a page when looking at its layout. The point of
using Lorem Ipsum is that it has a more-or-less normal distribution of
letters, as opposed to using ``Content here, content here'', making it
look like readable English. Many desktop publishing packages and web
page editors now use Lorem Ipsum as their default model text, and a
search for ``lorem ipsum'' will uncover many web sites still in their
infancy. Various versions have evolved over the years, sometimes by
accident, sometimes on purpose (injected humour and the like). 

\section{Onde se Podem Encontrar Exemplos?}

There are many variations of passages of Lorem Ipsum available, but
the majority have suffered alteration in some form, by injected
humour, or randomised words which don't look even slightly
believable. If you are going to use a passage of Lorem Ipsum, you need
to be sure there isn't anything embarrassing hidden in the middle of
text. All the Lorem Ipsum generators on the Internet tend to repeat
predefined chunks as necessary, making this the first true generator
on the Internet. It uses a dictionary of over 200 Latin words,
combined with a handful of model sentence structures, to generate
Lorem Ipsum which looks reasonable. The generated Lorem Ipsum is
therefore always free from repetition, injected humour, or
non-characteristic words etc. 


%% Index
%% Uncomment next command if index is required
%% don't forget to run ``makeindex pdis-en'' command
%\PrintIndex

\end{document}
